\documentclass[a4paper]{article}

\usepackage[utf8x]{inputenc}
\usepackage[T1]{fontenc}
\usepackage[portuguese]{babel}
\usepackage[lang=PT,box,completemulti]{automultiplechoice}

% "lang=PT" fui eu que coloquei. O "portuguese babel" acho que nem funciona.

\usepackage{multicol}
 
\begin{document}

\onecopy{2}{    

%%% beginning of the test sheet header:    

\noindent{\bf QCM  \hfill TESTE}

\vspace*{.5cm}
\begin{minipage}{.4\linewidth}
\centering\large\bf Teste\\ Exame feito em 27/10/2020\end{minipage}
\namefield{\fbox{    
                \begin{minipage}{.5\linewidth}
                  Nome:

                  \vspace*{.5cm}\dotfill
                  \vspace*{1mm}
                \end{minipage}
         }}

\begin{center}\em
Duração: 10 minutes.

No documento allowed. The use of electronic calculators is forbidden.

Questões using the sign \multiSymbole{} may have zero, one or several correct answers.
Other questions have a single correct answer.
Negative points may be attributed to \emph{very bad} answers.
\end{center}
\vspace{1ex}

%%% end of the header

\begin{question}{Questao-3.1-Halliday-4ed} % Ambiente para apenas uma questão correta. A segunda Chave é um marcador.
  A respeito da Among the following persons, which one has ever been a President of the French Republic?
  \begin{choices}
    \correctchoice{René Coty}
    \wrongchoice{Alain Prost}
    \wrongchoice{Marcel Proust}
    \wrongchoice{Claude Monet}
  \end{choices}
\end{question}


\begin{questionmult}{pref} % Ambiente para mais de uma questão correta. A segunda Chave é um marcador.
  Among the following cities, which ones are French prefectures?
  \begin{choices}
    \correctchoice{Poitiers}
    \wrongchoice{Sainte-Menehould}
    \correctchoice{Avignon}
  \end{choices}
\end{questionmult}




\begin{question}{Questao-pagina-100-Gaspar} % Ambiente para apenas uma questão correta. A segunda Chave é um marcador.
% Questão obtida o livro Gaspar, página 100
(Enem) uma empresa de transporte precisa efetuar a entrega de uma encomenda o mais breve possível. Para tanto, 
a equipe de logística analisa o trajeto desde a empresa até o local da entrega. Ela verifica que o trajeto apresenta
dois trechos de distâncias diferentes e velocidades máximas permitidas diferentes. No primeiro trecho, a velocidade
 máxima permitida é de 80 km/h e a distância a ser percorrida é de 80 km. No segundo trecho, cujo comprimento 
vale 60 km, a velocidade máxima permitida é 120 km/h. Supondo que as condições de trânsito sejam favoráveis
para que o veículo da empresa ande continuamente na velocidade máxima permitida, qual será o tempo necessário, 
em horas, para a realização da entrega?
\begin{choices}
\wrongchoice{0,7}
\wrongchoice{1,4}
\correctchoice{1,5}
\wrongchoice{2,0}
\wrongchoice{3,0}
\end{choices}
\end{question}



% \AMCaddpagesto{3} 

}   

\end{document}
